%%%%%%%%%%%%%%%%%%%%%%%%%%%%%%%%%%%%%%%%%
% Beamer Presentation
% LaTeX Template
% Version 1.0 (10/11/12)
%
% This template has been downloaded from:
% http://www.LaTeXTemplates.com
%
% License:
% CC BY-NC-SA 3.0 (http://creativecommons.org/licenses/by-nc-sa/3.0/)
%
%%%%%%%%%%%%%%%%%%%%%%%%%%%%%%%%%%%%%%%%%

%----------------------------------------------------------------------------------------
%	PACKAGES AND THEMES
%----------------------------------------------------------------------------------------

\documentclass{beamer}

\usepackage{ctex}

\mode<presentation> {
	
	% The Beamer class comes with a number of default slide themes
	% which change the colors and layouts of slides. Below this is a list
	% of all the themes, uncomment each in turn to see what they look like.
	
	%\usetheme{default}
	%\usetheme{AnnArbor}
	%\usetheme{Antibes}
	%\usetheme{Bergen}
	\usetheme{Berkeley}
	%\usetheme{Berlin}
	%\usetheme{Boadilla}
	%\usetheme{CambridgeUS}
	%\usetheme{Copenhagen}
	%\usetheme{Darmstadt}
	%\usetheme{Dresden}
	%\usetheme{Frankfurt}
	%\usetheme{Goettingen}
	%\usetheme{Hannover}
	%\usetheme{Ilmenau}
	%\usetheme{JuanLesPins}
	%\usetheme{Luebeck}
	%\usetheme{Madrid}
	%\usetheme{Malmoe}
	%\usetheme{Marburg}
	%\usetheme{Montpellier}
	%\usetheme{PaloAlto}
	%\usetheme{Pittsburgh}
	%\usetheme{Rochester}
	%\usetheme{Singapore}
	%\usetheme{Szeged}
	%\usetheme{Warsaw}
	
	% As well as themes, the Beamer class has a number of color themes
	% for any slide theme. Uncomment each of these in turn to see how it
	% changes the colors of your current slide theme.
	
	%\usecolortheme{albatross}
	%\usecolortheme{beaver}
	%\usecolortheme{beetle}
	%\usecolortheme{crane}
	%\usecolortheme{dolphin}
	%\usecolortheme{dove}
	%\usecolortheme{fly}
	%\usecolortheme{lily}
	%\usecolortheme{orchid}
	%\usecolortheme{rose}
	%\usecolortheme{seagull}
	%\usecolortheme{seahorse}
	%\usecolortheme{whale}
	%\usecolortheme{wolverine}
	
	%\setbeamertemplate{footline} % To remove the footer line in all slides uncomment this line
	%\setbeamertemplate{footline}[page number] % To replace the footer line in all slides with a simple slide count uncomment this line
	
	%\setbeamertemplate{navigation symbols}{} % To remove the navigation symbols from the bottom of all slides uncomment this line
}

\usepackage{graphicx} % Allows including images
\usepackage{booktabs} % Allows the use of \toprule, \midrule and \bottomrule in tables
\usepackage{subfigure}

%----------------------------------------------------------------------------------------
%	TITLE PAGE
%----------------------------------------------------------------------------------------

\title[Cupt2020]{Cupt2020 - 13组} % The short title appears at the bottom of every slide, the full title is only on the title page

\author{13组} % Your name
\institute[UCLA] % Your institution as it will appear on the bottom of every slide, may be shorthand to save space
{
	University of California \\ % Your institution for the title page
	\medskip
	\textit{annan@shanghaitech.edu.cn} % Your email address
}

\begin{document}
	
	\begin{frame}
		\titlepage % Print the title page as the first slide
	\end{frame}
	
	\begin{frame}
		\frametitle{目录} % Table of contents slide, comment this block out to remove it
		\tableofcontents % Throughout your presentation, if you choose to use \section{} and \subsection{} commands, these will automatically be printed on this slide as an overview of your presentation
	\end{frame}
	
	%----------------------------------------------------------------------------------------
	%	PRESENTATION SLIDES
	%----------------------------------------------------------------------------------------
	
	%------------------------------------------------
	\section{\fangsong{实验介绍}} % Sections can be created in order to organize your presentation into discrete blocks, all sections and subsections are automatically printed in the table of contents as an overview of the talk
	%------------------------------------------------
	\subsection{\fangsong{第九题简介}}
	\begin{frame}
		\frametitle{\fangsong{第九题简介}}
		Under certain circumstances, the 'flea' of a magnetic stirrer can rise up and levitate stably in a viscous fluid during stirring. Investigate the origins of the dynamic stabilization of the 'flea' and how it depends on the relevant parameters.
		\\\fangsong{在一定条件下,磁力搅拌器的“搅拌子”在搅拌过程中可以在粘性流体中稳定地上升和悬浮。研究“搅拌子”动态稳定的原因及其如何依赖于相关参数。}
	\end{frame}
	
	%------------------------------------------------
	\subsection{\fangsong{基本量测量}}
	\begin{frame}
		\frametitle{\fangsong{基本量测量}}
		\begin{itemize}
			\item\heiti{{实验仪器和材料}}\\
			\fangsong{条形搅拌子,甘油,水,量筒(100ml,1000ml),磁铁,结晶皿,滴管,小钢珠,电子天平,秒表,照相机,LED灯}
			\item\heiti{{实验过程}}\\
			\fangsong{a) 流体密度的测量\\
				b) 流体粘滞系数的测量\\
				由于实验条件的限制,我们没有合适的测量流体粘滞系数的仪器,采用落球法进行测量。\\}
		\end{itemize}
		\begin{figure}[H]
			\includegraphics[width=0.4\textwidth]{落球法}
			\caption{\fangsong{落球法测量流体粘滞系数}}
		\end{figure}
	\end{frame}
	
	\begin{frame}
		\frametitle{\fangsong{基本量测量}}
		\begin{itemize}
			\item\heiti{{实验过程}}\\
			\fangsong{c) 流体密度的测量\\
				将磁子放在1000ml量筒中,量筒中盛放有一定量的实验用的甘油水溶液。在量筒外使用两块磁铁,将磁子吸至液体表面并控制磁子在量筒中保持水平。然后释放磁铁,使之自然下落。使用固定的照相机记录整个过程,使用tracker追踪磁子。	
			}
		\end{itemize}
		\begin{figure}[H]
			\includegraphics[width=0.2\textwidth]{磁子收尾速度的测量}
			\caption{\fangsong{磁子收尾速度的测量}}
		\end{figure}
	\end{frame}
	
	\subsection{\fangsong{定量分析}}
	\begin{frame}
		\frametitle{\fangsong{定量分析}}
		\begin{itemize}
			\item\heiti{{实验仪器和材料}}\\
			\fangsong{磁力搅拌器,条形搅拌子,甘油,水,结晶皿,玻璃杯,秒表,照相机,记号笔}
		\end{itemize}
	\end{frame}
	
	\begin{frame}
		\frametitle{\fangsong{定量分析}}
		\begin{itemize}
			\item\heiti{{实验过程}}\\
			\fangsong{a) 磁子高度与搅拌器转速的关系\\
				在玻璃杯内倒入甘油的水溶液,并在中央放入磁子。将搅拌器转速调至最高转速(1500r/min)后打开,使搅拌器转速从0开始随时间不断增加。使用固定的照相机记录全过程,使用tracker追踪磁子。
			}
		\end{itemize}
		\begin{figure}[H]
			\includegraphics[width=0.2\textwidth]{磁子高度与搅拌器转速定量关系测量的实验过程}
			\caption{\fangsong{磁子高度与搅拌器转速定量关系测量的实验过程}}
		\end{figure}
	\end{frame}
	
	\begin{frame}
		\frametitle{\fangsong{定量分析}}
		\begin{itemize}
			\item\heiti{{实验过程}}\\
			\fangsong{b) 磁子高度变化与时间的关系\\
				在磁子外套焊有小锡球的小环以便tracker准确追踪,将其放在倒有甘油水溶液的结晶皿中央。调节磁力搅拌器的转速分别为1000、1100、1200、1300、1400、1500r/min,待磁子稳定后使用固定的照相机慢镜头拍摄,使用tracker追踪磁子下方的小锡球。
			}
		\end{itemize}
		\begin{figure}[H]
			\includegraphics[width=0.4\textwidth]{套有小环的磁子}
			\caption{\fangsong{套有小环的磁子}}
		\end{figure}
	\end{frame}
	
	\begin{frame}
		\frametitle{\fangsong{定量分析}}
		\begin{itemize}
			\item\heiti{{实验过程}}\\
			\fangsong{c) 磁子旋转的角度与时间的关系\\
				在磁子一端用黑色记号笔标记以便tracker准确追踪,将其放在倒有甘油水溶液的结晶皿中央。调节磁力搅拌器的转速分别为1100、1200、1300、1400、1500r/min,待磁子稳定后使用固定的照相机慢镜头拍摄,使用tracker追踪磁子的一端。
			}
		\end{itemize}
		\begin{figure}[H]
			\includegraphics[width=0.3\textwidth]{磁子旋转的角度与时间定量关系测量的实验过程}
			\caption{\fangsong{磁子旋转的角度与时间定量关系测量的实验过程}}
		\end{figure}
	\end{frame}
	
	
	%------------------------------------------------
	\subsection{\fangsong{实验结果}}
	\begin{frame}
		\frametitle{\fangsong{实验结果}}
		\begin{itemize}
			\item\heiti{{预实验结果}}\\
			\fangsong{\qquad 可以观察到题目中所描述的现象:磁子在甘油中稳定地悬浮并伴有转动和上下振动。}
			\vspace{0.25cm}
			\begin{figure}
				\centering
				\includegraphics[height=4.5cm]{img/yushiyan.png}
			\end{figure}
		\end{itemize}
	\end{frame}
	
	\begin{frame}
		\frametitle{\fangsong{实验结果}}
		\begin{itemize}
			\item\heiti{关于假设的实验结果}\\
			\fangsong{a)观察到明显的爬杆效应。但此时在流体中放入磁子,它并没有像我们想象的那样受到爬杆效应影响而悬浮,而是可以在容器底部照常高速旋转。于是,我们可以初步否定此现象与爬杆效应的关系。}
			\begin{figure}[htbp]
				\vspace{-0.1cm}
				\centering
				\subfigure{
					\centering
					\includegraphics[height=2.5cm]{img/pagan.png}
				}
				\subfigure{
					\centering
					\includegraphics[height=2.5cm]{img/paganzoom.png}
				}
			\end{figure}
			\fangsong{b)在一定的转速下,磁子可以发生悬浮。虽然不是十分稳定,但可以确定是由于水的粘滞阻力小,微小的扰动就会破坏磁子的悬浮状态。这可以进一步验证此现象与爬杆效应无关。初步推断这一现象是由于搅拌器与搅拌子磁场不同步所致。}
		\end{itemize}
	
	\end{frame}
	
	\begin{frame}
		\frametitle{\fangsong{实验结果}}
		\begin{itemize}
			\item\heiti{基本量测量的结果}\\
			\fangsong{a) 实验用的甘油水溶液的密度1.302g/ml\\
				b) 实验用的甘油水溶液的粘度 \\
				c) 磁子的收尾速度 0.1318m/s
			}
		\end{itemize}
	\end{frame}
	
	\begin{frame}
		\frametitle{\fangsong{实验结果}}
		\begin{itemize}
			\item\heiti{定量分析}\\
			\fangsong{a) 磁子高度与搅拌器转速的关系}\\
			\qquad 我们得到了磁子高度与搅拌器转速的关系图。
			\begin{figure}
				\vspace{-0.3cm}
				\centering
				\includegraphics[height=3cm]{img/h-t.png}
			\end{figure}
			\fangsong{\qquad 可以观察到起初磁子照常在结晶皿底部高速旋转,搅拌器达到一定转速时,磁子突然跳起,随后保持悬浮,并不断伴有上下振动。从图中可以看出,随着搅拌器转速不断增加,磁子悬浮的高度(上下振动的平衡位置的高度)有缓慢下降。}
		\end{itemize}
	\end{frame}
	
	\begin{frame}
		\frametitle{\fangsong{实验结果}}
		\begin{itemize}
			\item\heiti{b) 磁子高度变化与时间的关系}
			\fangsong{我们得到了一组不同转速下磁子高度变化与时间的关系图。}
			\begin{figure}
				\centering
				\includegraphics[height=4cm]{img/deltah-t.png}
			\end{figure}
			\fangsong{磁子高度变化与时间的关系图(搅拌器转速依次为1100、1200、1300、1400、1500r/min)\\
			从图中可以得出,在不同转速下,磁子高度随时间成正弦变化,即磁子在粘性流体中上下简谐振动。}
		\end{itemize}
	\end{frame}
	
	\begin{frame}
		\frametitle{\fangsong{实验结果}}
		\begin{itemize}
			\item\heiti{c) 磁子旋转的角度与时间的关系}
			\fangsong{我们得到了一组不同转速下磁子旋转的角度与时间的关系图。同时,我们使用matlab对微分方程求解,同样可以得到一组同样条件下磁子旋转的角度与时间的关系图,将二者进行对比。}
			\begin{figure}
				\centering
				\includegraphics[height=2.5cm]{img/deltatheta-t.png}
			\end{figure}
			\fangsong{(图中上面为matlab运算得到的理论图像;下面为实验得到的图像。纵轴单位均为弧度)\\
			不同转速下,磁子旋转的角度随时间变化关系为一次函数与正弦函数的叠加,即磁子在流体中旋转一个大角度后会反向旋转一个小角度,再正向旋转一个大角度,如此往复地旋转。与参考文献中的结果相似。}
		\end{itemize}
	\end{frame}
	
	%------------------------------------------------
	\section{\fangsong{理论部分}}
	%------------------------------------------------
	
	\section{\fangsong{数据分析}}
	\section{\fangsong{结论}}
	%------------------------------------------------
	
	\begin{frame}
		\Huge{\centerline{\fangsong{砍人大赛:街头霸王}}}
	\end{frame}
	
	%----------------------------------------------------------------------------------------
	
\end{document} 